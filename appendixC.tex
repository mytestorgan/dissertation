\fontsize{10pt}{10pt}\selectfont
\begin{verbatim}
     1  #include <fstream>
     2  #include "vector.h"
     3  #include <math.h>
     4
     5  #include "ns3/core-module.h"
     6  #include "ns3/network-module.h"
     7  #include "ns3/internet-module.h"
     8  #include "ns3/mobility-module.h"
     9  #include "ns3/lte-module.h"
    10  #include "ns3/applications-module.h"
    11  #include "ns3/point-to-point-module.h"
    12  #include "ns3/config-store-module.h"
    13  using namespace ns3;
    14
    15  vector<float> send;
    16  vector<float> receive;
    17
    18
    19  static void
    20  Ipv4L3ProtocolRxSinkWithoutContext (Ptr<const Packet> packet, const Addr
ess &ad)  {
    21    
    22      receive.insert(packet->GetUid()/2, Simulator::Now ().GetSeconds ());
    23    //std::cout << 'r'  << Simulator::Now ().GetSeconds () << " " << packe
t->GetUid()/2 << std::endl;
    24  }
    25  static void
    26  Ipv4L3ProtocolTxSinkWithoutContext (Ptr<const Packet> packet)  {
    27    
    28      send.insert(packet->GetUid()/2, Simulator::Now ().GetSeconds ());
    29    //std::cout << 's' << Simulator::Now ().GetSeconds () << " " << packet
->GetUid()/2 << std::endl;
    30  }
    31
    32
    33  void 
    34  NotifyConnectionEstablishedUe (std::string context, 
    35                                 uint64_t imsi, 
    36                                 uint16_t cellid, 
    37                                 uint16_t rnti)
    38  {
    39    std::cout << context 
    40              << " UE IMSI " << imsi 
    41              << ": connected to CellId " << cellid 
    42              << " with RNTI " << rnti 
    43              << std::endl;
    44  }
    45
    46  void 
    47  NotifyHandoverStartUe (std::string context, 
    48                         uint64_t imsi, 
    49                         uint16_t cellid, 
    50                         uint16_t rnti, 
    51                         uint16_t targetCellId)
    52  {
    53    std::cout << context 
    54              << " UE IMSI " << imsi 
    55              << ": previously connected to CellId " << cellid 
    56              << " with RNTI " << rnti 
    57              << ", doing handover to CellId " << targetCellId 
    58              << std::endl;
    59  }
    60
    61  void 
    62  NotifyHandoverEndOkUe (std::string context, 
    63                         uint64_t imsi, 
    64                         uint16_t cellid, 
    65                         uint16_t rnti)
    66  {
    67    std::cout << context 
    68              << " UE IMSI " << imsi 
    69              << ": successful handover to CellId " << cellid 
    70              << " with RNTI " << rnti 
    71              << std::endl;
    72  }
    73
    74  void 
    75  NotifyConnectionEstablishedEnb (std::string context, 
    76                                  uint64_t imsi, 
    77                                  uint16_t cellid, 
    78                                  uint16_t rnti)
    79  {
    80    std::cout << context 
    81              << " eNB CellId " << cellid 
    82              << ": successful connection of UE with IMSI " << imsi 
    83              << " RNTI " << rnti 
    84              << std::endl;
    85  }
    86
    87  void 
    88  NotifyHandoverStartEnb (std::string context, 
    89                          uint64_t imsi, 
    90                          uint16_t cellid, 
    91                          uint16_t rnti, 
    92                          uint16_t targetCellId)
    93  {
    94    std::cout << context 
    95              << " eNB CellId " << cellid 
    96              << ": start handover of UE with IMSI " << imsi 
    97              << " RNTI " << rnti 
    98              << " to CellId " << targetCellId 
    99              << std::endl;
   100  }
   101
   102  void 
   103  NotifyHandoverEndOkEnb (std::string context, 
   104                          uint64_t imsi, 
   105                          uint16_t cellid, 
   106                          uint16_t rnti)
   107  {
   108    std::cout << context 
   109              << " eNB CellId " << cellid 
   110              << ": completed handover of UE with IMSI " << imsi 
   111              << " RNTI " << rnti 
   112              << std::endl;
   113  }
   114   
   115  NS_LOG_COMPONENT_DEFINE ("EpcX2HandoverExample");
   116  int
   117  main (int argc, char *argv[])
   118  {
   119    uint16_t numberOfUes = 1;
   120    uint16_t numberOfEnbs = 2;
   121    double simTime = 2.300;
   122    double distance = 1000.0;
   123
   124    // Command line arguments
   125    CommandLine cmd;
   126    cmd.AddValue("numberOfUes", "Number of UEs", numberOfUes);
   127    cmd.AddValue("numberOfEnbs", "Number of eNodeBs", numberOfEnbs);
   128    cmd.AddValue("simTime", "Total duration of the simulation (in seconds)
",simTime);
   129    cmd.Parse(argc, argv);
   130
   131    Ptr<LteHelper> lteHelper = CreateObject<LteHelper> ();
   132    Ptr<EpcHelper> epcHelper = CreateObject<EpcHelper> ();
   133    lteHelper->SetEpcHelper (epcHelper);
   134    lteHelper->SetSchedulerType("ns3::RrFfMacScheduler");
   135
   136    Ptr<Node> pgw = epcHelper->GetPgwNode ();
   137
   138    // Create a single RemoteHost
   139    NodeContainer remoteHostContainer;
   140    remoteHostContainer.Create (1);
   141    Ptr<Node> remoteHost = remoteHostContainer.Get (0);
   142    InternetStackHelper internet;
   143    internet.Install (remoteHostContainer);
   144
   145    // Create the Internet
   146    PointToPointHelper p2ph;
   147    p2ph.SetDeviceAttribute ("DataRate", DataRateValue (DataRate ("100Gb/s
")));
   148    p2ph.SetDeviceAttribute ("Mtu", UintegerValue (1500));
   149    p2ph.SetChannelAttribute ("Delay", TimeValue (Seconds (0.010)));
   150    NetDeviceContainer internetDevices = p2ph.Install (pgw, remoteHost);
   151    Ipv4AddressHelper ipv4h;
   152    ipv4h.SetBase ("1.0.0.0", "255.0.0.0");
   153    Ipv4InterfaceContainer internetIpIfaces = ipv4h.Assign (internetDevice
s);
   154
   155    // Routing of the Internet Host (towards the LTE network)
   156    Ipv4StaticRoutingHelper ipv4RoutingHelper;
   157    Ptr<Ipv4StaticRouting> remoteHostStaticRouting = ipv4RoutingHelper.Get
StaticRouting (remoteHost->GetObject<Ipv4> ());
   158    // interface 0 is localhost, 1 is the p2p device
   159    remoteHostStaticRouting->AddNetworkRouteTo (Ipv4Address ("7.0.0.0"), I
pv4Mask ("255.0.0.0"), 1);
   160
   161    NodeContainer ueNodes;
   162    NodeContainer enbNodes;
   163    enbNodes.Create(numberOfEnbs);
   164    ueNodes.Create(numberOfUes);
   165
   166    // Install Mobility Model
   167    Ptr<ListPositionAllocator> positionAlloc = CreateObject<ListPositionAl
locator> ();
   168    
   169    positionAlloc->Add (Vector(distance, 300, 0));
   170    positionAlloc->Add (Vector(distance, 0, 0));
   171    positionAlloc->Add (Vector(0, 150, 0));
   172    
   173    MobilityHelper mobility;
   174    mobility.SetMobilityModel("ns3::ConstantPositionMobilityModel");
   175    mobility.SetPositionAllocator(positionAlloc);
   176    mobility.Install(enbNodes);
   177    mobility.Install(ueNodes);
   178
   179    // Install LTE Devices in eNB and UEs
   180    NetDeviceContainer enbLteDevs = lteHelper->InstallEnbDevice (enbNodes)
;
   181    NetDeviceContainer ueLteDevs = lteHelper->InstallUeDevice (ueNodes);
   182
   183    // Install the IP stack on the UEs
   184    internet.Install (ueNodes);
   185    Ipv4InterfaceContainer ueIpIfaces;
   186    ueIpIfaces = epcHelper->AssignUeIpv4Address (NetDeviceContainer (ueLte
Devs));
   187    // Assign IP address to UEs, and install applications
   188    for (uint32_t u = 0; u < ueNodes.GetN (); ++u)
   189      {
   190        Ptr<Node> ueNode = ueNodes.Get (u);
   191        // Set the default gateway for the UE
   192        Ptr<Ipv4StaticRouting> ueStaticRouting = ipv4RoutingHelper.GetStat
icRouting (ueNode->GetObject<Ipv4> ());
   193        ueStaticRouting->SetDefaultRoute (epcHelper->GetUeDefaultGatewayAd
dress (), 1);
   194      }
   195
   196    // Attach all UEs to the first eNodeB
   197    for (uint16_t i = 0; i < numberOfUes; i++)
   198      {
   199        lteHelper->Attach (ueLteDevs.Get(i), enbLteDevs.Get(0));
   200      }
   201
   202    NS_LOG_LOGIC ("setting up applications");
   203      
   204    for (uint32_t u = 0; u < numberOfUes; ++u)
   205      {
   206        Ptr<Node> ue = ueNodes.Get (u);
   207        // Set the default gateway for the UE
   208        Ptr<Ipv4StaticRouting> ueStaticRouting = ipv4RoutingHelper.GetStat
icRouting (ue->GetObject<Ipv4> ());
   209        ueStaticRouting->SetDefaultRoute (epcHelper->GetUeDefaultGatewayAd
dress (), 1);
   210
   211      //Install Application
   212    uint16_t port = 1234;      
   213    Config::SetDefault ("ns3::OnOffApplication::PacketSize", UintegerValue
 (512));
   214    Config::SetDefault ("ns3::OnOffApplication::DataRate", StringValue ("3
000kb/s"));
   215    
   216    OnOffHelper clientHelper ("ns3::UdpSocketFactory", InetSocketAddress("
1.0.0.2", port));
   217    clientHelper.SetAttribute ("OnTime", StringValue ("ns3::ConstantRandom
Variable[Constant=1]"));
   218    clientHelper.SetAttribute ("OffTime", StringValue ("ns3::ConstantRando
mVariable[Constant=0]"));
   219    ApplicationContainer apps = clientHelper.Install(ueNodes.Get(0));
   220
   221    PacketSinkHelper sink(  "ns3::UdpSocketFactory", InetSocketAddress("1.
0.0.2", port));
   222    sink.Install(remoteHostContainer.Get (0));
   223
   224    apps.Start (Seconds (0.1));
   225    apps.Stop (Seconds (2));
   226      }
   227
   228    // Add X2 inteface
   229    lteHelper->AddX2Interface (enbNodes);
   230
   231    // X2-based Handover
   232    lteHelper->HandoverRequest (Seconds (0.100), ueLteDevs.Get (0), enbLte
Devs.Get (0), enbLteDevs.Get (1));
   233    lteHelper->HandoverRequest (Seconds (0.500), ueLteDevs.Get (0), enbLte
Devs.Get (1), enbLteDevs.Get (0));
   234    lteHelper->HandoverRequest (Seconds (0.700), ueLteDevs.Get (0), enbLte
Devs.Get (0), enbLteDevs.Get (1));
   235    lteHelper->HandoverRequest (Seconds (1.00), ueLteDevs.Get (0), enbLteD
evs.Get (1), enbLteDevs.Get (0));
   236    
   237    // Uncomment to enable PCAP tracing
   238    //p2ph.EnablePcapAll("lena-x2-handover");
   239
   240    lteHelper->EnableMacTraces ();
   241    lteHelper->EnableRlcTraces ();
   242    lteHelper->EnablePdcpTraces ();
   243    Ptr<RadioBearerStatsCalculator> rlcStats = lteHelper->GetRlcStats ();
   244    rlcStats->SetAttribute ("EpochDuration", TimeValue (Seconds (0.05)));
   245    Ptr<RadioBearerStatsCalculator> pdcpStats = lteHelper->GetPdcpStats ()
;
   246    pdcpStats->SetAttribute ("EpochDuration", TimeValue (Seconds (0.05)));
   247
   248  //   connect custom trace sinks for RRC connection establishment and han
dover notification
   249    Config::Connect ("/NodeList/*/DeviceList/*/LteEnbRrc/ConnectionEstabli
shed",
   250                     MakeCallback (&NotifyConnectionEstablishedEnb));
   251    Config::Connect ("/NodeList/*/DeviceList/*/LteUeRrc/ConnectionEstablis
hed",
   252                     MakeCallback (&NotifyConnectionEstablishedUe));
   253    Config::Connect ("/NodeList/*/DeviceList/*/LteEnbRrc/HandoverStart",
   254                     MakeCallback (&NotifyHandoverStartEnb));
   255    Config::Connect ("/NodeList/*/DeviceList/*/LteUeRrc/HandoverStart",
   256                     MakeCallback (&NotifyHandoverStartUe));
   257    Config::Connect ("/NodeList/*/DeviceList/*/LteEnbRrc/HandoverEndOk",
   258                     MakeCallback (&NotifyHandoverEndOkEnb));
   259    Config::Connect ("/NodeList/*/DeviceList/*/LteUeRrc/HandoverEndOk",
   260                     MakeCallback (&NotifyHandoverEndOkUe));
   261    Config::ConnectWithoutContext ("/NodeList/*/ApplicationList/*/$ns3::Pa
cketSink/Rx",
   262                                   MakeCallback (&Ipv4L3ProtocolRxSinkWith
outContext));
   263    Config::ConnectWithoutContext ("/NodeList/*/ApplicationList/*/$ns3::On
OffApplication/Tx",
   264                                   MakeCallback (&Ipv4L3ProtocolTxSinkWith
outContext));
   265    Simulator::Stop(Seconds(simTime));
   266    Simulator::Run();
   267
   268    Simulator::Destroy();
   269    std::ofstream myfile;
   270    myfile.open ("scratch/hand.out");
   271    
   272    float sum=0.0;
   273    int num_delay=0;
   274    float delay;
   275    float countDrop;
   276    float perDrop;
   277    for(int i =0; i<send.nItem-1; i++){
   278        delay= receive[i]-send[i];
   279        if ((send[i]!=-1)and(receive[i]!=-1)and(delay<5.0)and(delay>0.0)){
   280            myfile<< i <<' '<<delay<<' '<<send[i]<<' '<<receive[i]<<std::e
ndl;
   281            sum+=receive[i]-send[i];
   282            num_delay++;
   283        }
   284        else{
   285            countDrop++;
   286        }
   287    }
   288    myfile.close();
   289    float middleDelay=sum/num_delay;
   290    sum=0.0;
   291    for(int i =0; i<send.nItem-1; i++){
   292         delay= receive[i]-send[i];
   293        if ((send[i]!=-1)and(receive[i]!=-1)and(delay<5.0)and(delay>0.0)){
   294            sum=sum+fabs(delay-middleDelay);
   295            
   296        }
   297    }
   298    float middleJitter=sum/num_delay;
   299    perDrop=(float)countDrop/(float)(send.nItem-1);
   300    std::cout<< distance << ' ' <<middleDelay << ' ' <<middleJitter << ' '
 << perDrop<<std::endl;
   301    return 0;
   302
   303  }
\end{verbatim}

\fontsize{12pt}{16pt}\selectfont



