\chapter{РАЗРАБОТКА АЛГОРИТМА УПРАВЛЕНИЯ БУФЕРОМ КОМПЕНСАЦИИ ДЖИТТЕРА} \label{chapt1}

Существующие телекоммуникационные системы активно используют различные методы управления: ситуационные, основанные на логике лиц, принимающих решение; автоматические; автоматизированные.
Вместе с тем, за последние годы все более процедур в телекоммуникационных сетях новых поколений осуществляется автоматически, с оптимизацией этих процедур, что позволяет за кратчайшее время получать наибольший эффект от управления данными процедурами.

В существующих технологиях большой удельный вес занимают методы управления основанные на принципах Понселе. Данные принцип основан на предположении о том, что любому обнаруженному возмущению находится адекватное управление, реагирующее на это возмущение.
Структурная схема управления, функционирующая по принципу Понселе, показана на рис. \ref{fig:ponsele}.


\begin{figure}[!h]

\centering
\begin{tikzpicture}
[node distance = 1cm, auto,
% STYLES
every node/.style={node distance=6cm},
% The force style is used to draw the forces' name
force/.style={rectangle, draw, fill=white!10, inner sep=5pt, text width=4cm, text badly centered, minimum height=1.2cm}] 


% Draw forces
\node [force] (resolving) {Принятое решение};
\node [force, right of=resolving] (object) {Управляемый объект};

 \draw [>=latex,->] (resolving) edge node {$u(t)$} (object);
\draw [>=latex,->] (object)    -- +(0,-2)  -| (resolving);
\node (ask) at (3, -2.5) {Подтверждение об исполнении};

\end{tikzpicture} 
\caption{Схема управления по возмущению (принцип Понселе)}
\label{fig:ponsele}
\end{figure}

Логично, что алгоритм компенсации джиттера должен быть реализован на основе подстройки линии задержки, как результат на отклонение оценки задержки. 
Из этого следует, что принцип управления Понселе для синтеза алгоритма управления буфером не подходит. 
Рассмотрим принцип Уатта, который основан на управлении по отклонению.
Данный принцип управления используется в тех устройствах, выходные сигналы которых имеют те или иные отклонения от средних  или типовых значений.
По сути принцип Уатта лежит в основе построения систем автоматического управления. Структурная схема устройства управления, построенного по принципу Уатта, представлена на рис. \ref{fig:uatta}


\begin{figure}[!h]

\centering
\begin{tikzpicture}
[node distance = 1cm, auto,
% STYLES
every node/.style={node distance=3cm},
% The force style is used to draw the forces' name
force/.style={rectangle, draw, fill=white!10, inner sep=5pt, text width=4cm, text badly centered, minimum height=1.2cm}] 


% Draw forces
\node [force] (system) {Управляемая система};
\node [force, below of=resolving] (man) {Устройство управления};
 \draw [>=latex,->] (resolving) edge node {$u(t)$} (object.190);
\draw [>=latex,->] (man.180)    -- +(-1,0)  |- (system.190);

\path[>=latex,->] (system.170){}+(-1,0) edge node {Вход} (system.170);
\path[>=latex,->] (system.0) edge node {Выход} (system.0){}+(2,0);


\end{tikzpicture} 
\caption{Схема управления по возмущению (принцип Понселе)}
\label{fig:ponsele}
\end{figure}