\chapter*{ВВЕДЕНИЕ}							% Заголовок

\textbf{Актуальность темы.} Согласно прогнозу \cite{ericsson} 60\% людей к концу 2018 года будут иметь покрытие LTE. Как мы видим беспроводные технологии все больше внедряются в наш повседневный мир. Такие техологии как WiFi, Bluetooth, GSM стали уже неотемлемой частью нашей жизни. 
Современные перспективные мобильные сети развиваются в направлении внедрения следуйщего поколения NGN (Next Generation Network). На данный момент основными представителями таких сетей являются WiMAX и LTE сети.

Основными преимуществами использования стандарта LTE является то что построенные на его основе, оптимизированы для передачиданных и реализованыв вде коммутации пакетов.
LTE не включает в себя домен коммутации пакетовдля предостовления услуг передачи речи.

Спрос на услуги мобильного широкополосного доступа растет и операторы запускаю высоко скоростные сети на основе LTE. Тем не менее, услуги передачи речи приносят около 70\% общего дохода операторови ясно, что эта функциональность должна быть реализована и в сетях LTE.

Сети LTE работают на стыке проводной и беспроводной сети. Практика показывает, что основные потери качественных характиристик обслуживания (QoS) происходят на границе различный сред передачи.
При передачи мультимедийной информации по комбинированным сетям с различных технологий передачи данных, важным является выполнение требований к качеству предостовления мультимедийной информации пользователю.
При этом для трафика реального времени такого, как трафик VoIP и видео звонков, важными являются такие сетевые характеристики:
задержка, число потерянных и поврежденных пакетов и джиттер задержки.
Согласно \cite{??} наибольший вклад в задержку и потери пакетов вносит не оптимальный буфер компенсации джиттера (буфер воспроизведения). До 40\% допустимой задержки, определенной в рекомендации \cite{G114}, может состовлять задержка буфера компенсации джиттера.
Еще одной проблемой является то, что оконечные устройства могут компенсироватьограниченный размер джиттера (порядка 50 мс).

Сдедвательно, актуальной является научная задача, которая состоит в разработке методов мониторингатекущего сетевого метода пердварительной компенсации джиттера на границах проводной и беспроводной сети.



Для~достижения поставленной цели необходимо было решить следующие задачи:
\begin{enumerate}
  \item Исследовать, разработать, вычислить и т.д. и т.п.
  \item Исследовать, разработать, вычислить и т.д. и т.п.
  \item Исследовать, разработать, вычислить и т.д. и т.п.
  \item Исследовать, разработать, вычислить и т.д. и т.п.
\end{enumerate}

\textbf{Основные положения, выносимые на~защиту:}
\begin{enumerate}
  \item Первое положение
  \item Второе положение
  \item Третье положение
  \item Четвертое положение
\end{enumerate}

\textbf{Научная новизна:}
\begin{enumerate}
  \item Впервые \ldots
  \item Впервые \ldots
  \item Было выполнено оригинальное исследование \ldots
\end{enumerate}

\textbf{Научная и практическая значимость} \ldots

\textbf{Степень достоверности} полученных результатов обеспечивается \ldots Результаты находятся в соответствии с результатами, полученными другими авторами.

\textbf{Апробация работы.}
Основные результаты работы докладывались~на:
перечисление основных конференций, симпозиумов и т.п.

\textbf{Личный вклад.} Автор принимал активное участие \ldots

\textbf{Публикации.} Основные результаты по теме диссертации изложены в ХХ печатных изданиях~\cite{bib1,bib2,bib3,bib4,bib5},
Х из которых изданы в журналах, рекомендованных ВАК~\cite{bib1,bib2,bib3}, 
ХХ --- в тезисах докладов~\cite{bib4,bib5}.

\textbf{Объем и структура работы.} Диссертация состоит из~введения, четырех глав, заключения и~двух приложений. Полный объем диссертации составляет ХХХ~страница с~ХХ~рисунками и~ХХ~таблицами. Список литературы содержит ХХХ~наименований.

\clearpage