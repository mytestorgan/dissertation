\chapter*{ВВЕДЕНИЕ}							% Заголовок

\textbf{Актуальность темы.} Как показывает практика, беспроводные технологии все больше внедряются в наш повседневный мир. Такие технологии как WiFi, Bluetooth, GSM стали уже неотъемлемой частью нашей жизни. 
Современные мобильные сети развиваются в направлении внедрения концепций следующего поколения NGN (Next Generation Network) \cite{zacon:net,zacon:tele}. На данный момент основными представителями таких сетей являются WiMAX и LTE сети.
Согласно прогнозу \cite{ericsson} 60\% людей к концу 2018 года будут иметь покрытие LTE.


Основными преимуществами использования стандарта LTE является то, что сети, построенные на его основе, оптимизированы для передачи данных и реализованы в виде коммутации пакетов 
и не включает в себя домен коммутации пакетов для предоставления услуг передачи речи.

Спрос на услуги мобильного широкополосного доступа растет, и операторы запускаю высокоскоростные сети на основе LTE. 
Тем не менее, услуги передачи речи приносят около 70\% общего дохода операторов и ясно, что эта функциональность должна быть реализована и в сетях LTE.

%усилить
Сети LTE работают на стыке проводной и беспроводной сети. %добавить да перед тем как сказать нет
Практика показывает, что основные потери качественных характеристик обслуживания (QoS) происходят на границе различных сред передачи.
При передаче мультимедийной информации по комбинированным сетям с различными технологиями передачи данных, важным является выполнение требований к качеству предоставления мультимедийной информации пользователю.
При этом для трафика реального времени такого, как трафик VoIP и видео связи, важными являются такие сетевые характеристики:
задержка, число потерянных и поврежденных пакетов и джиттер задержки.
Согласно \cite{rokovoy} наибольший вклад в задержку и потери пакетов вносит неоптимальный буфер компенсации джиттера (буфер воспроизведения). До 40\% допустимой задержки, определенной в рекомендации \cite{G114}, может составлять задержка буфера компенсации джиттера.
Еще одной проблемой является то, что оконечные устройства могут компенсировать ограниченный размер джиттера (порядка~50~мс).

Следовательно, актуальной является научная задача, которая состоит в разработке методов предварительной компенсации джиттера на границах проводной и беспроводной сети.


\textbf{Связь работы с научными программами, планами и темами.} Диссертационная работа связана с реализацией основных положений <<Концепции национальной информационной политики>>, 
<<Концепции Национальной программы информатизации>>, <<Основных принципов развития информационного общества в Украине на 2007 - 2015 года>> и <<Концепции конвергенции телефонных сетей и сетей с пакетной коммутацией в Украине>>.
Результаты работы использованы при выполнении научно-исследовательской работы №1261-1 <<Методи підвищення продуктивності безпроводових мереж наступного покоління>> (№ госрегистрации 0111U002627), которая выполнялась кафедрой телекоммуникационных систем Харьковского национального университета радиоэлектроники. В указанной научно-исследовательской работе диссертант был исполнителем.

\textbf{Цель и задача исследования} состоит в повышении качества обслуживания в гибридных сетях, которые содержат мобильную и стационарную компоненту.

В ходе решения научной задачи сформулированы и решены частные задачи исследования:
\begin{enumerate}
  \item Провести анализ статистических характеристик джиттера в стационарных и беспроводных сетях.
  \item Определить основные причины формирования джиттера.
  \item Определить статистические характеристики нестационарности джиттера и произвести классификацию нестационарных явлений задержки.
  \item Обосновать и разработать математическую модель джиттера, позволяющую отображать динамику изменений состояний сетевой задержки.
  \item Разработать алгоритмы стохастической оценки параметров джиттера и управления с целью его минимизации.
  \item Разработать практические предложения по выбору параметров и мест установки агента минимизации джиттера на границе стационарной и мобильной сети.
\end{enumerate}

{\itshape Объект исследования:} процесс передачи трафика реального времени через гибридные сети.

{\itshape Предмет исследования:} метод повышения качества обслуживания на основе потоковых агентов на стыке мобильных и стацонарных сетей.

{\itshape Методы исследования.} 
В ходе разработки алгоритма статистической оценки параметров джиттера были использованы методы теории связи, математической статистики, теории вероятности случайных процессов, 
теории решений, непараметрические методы обработки, робастный фильтр Калмана-Бьюси. 
Для разработки математической модели джиттера был использован аппарат теории выбросов. 
В ходе проведения оценки эффективности использовались методы имитационного моделирования.


\textbf{Научная новизна полученных результатов.} 
\begin{enumerate}
  \item В результате анализа состояния составных каналов связи, включая мобильную и стационарную компоненту, выявлены причины возникновения нестационарностей и большого разброса параметров джиттера.
  Проанализированы механизмы формирования джиттера в гибридных сетях, получены статистические данные характеристик джиттера.
  
  %Проанализированы причины возникновения и механизмы формирования джиттера в гибридных телекоммуникационных сетях, 
  %что показало на необходимость дополнительного метода компенсации джиттера на стыке двух сред мобильной и стационарной компоненты.
 

  \item Разработана более адекватная общая, по сравнению с известными, нестационарная математическая модель задержки прибытия пакетов,
  %учитывающая засоренное представление наблюдаемого процесса, позволяющая в отличии от известных моделей, учитывать скачки и выбросы.
  позволяющая учитывать засоренность представления наблюдаемого процесса случайными выбросами и скачками.
  \item 
  Разработаны новый адаптивный метод компенсации джиттера на базе робастных процедур инвариантных к распределению вероятностей процесса задержки.
  %Проанализированы рекурсивные адаптивные методы компенсации джиттера, что показало на необходимость поиска решений среди непараметрических робастных алгоритмов. 
  % Впервые для компенсации джиттера предложено использовать робастный фильтр Калмана-Бьюси.
  %Разработан робастный, инвариантный к распределению скачков и выбросов метод управления буфером компенсации джиттера позволяющий обеспечить на границе стационарной и мобильной сети дополнительную компенсацию, чем обеспечивает высокий уровень качества обслуживания.
  \item 
  Разработаны новые рекомендации по применению буфера компенсации джиттера в сетях LTE на основе потоковых агентов, устанавливаемых на границе проводной и беспроводной сети.
  %Предложены рекомендации по практическому применению буфера компенсации джиттера в сетях LTE на основе потоковых агентов, устанавливаемых на границе проводной и беспроводной сети.
\end{enumerate}

\textbf{Практическое значение полученных результатов.} Полученные научные результаты имеют практическое значение, 
поскольку они ориентированы на дальнейшее внедрение в реальные системы связи, 
в частности, в диссертационных исследованиях предложен новый метод предварительной компенсации джиттера на границе проводной и беспроводной сети на основе потовых агентов,
что позволяет обеспечить повышение качества передачи речевого трафика в гибридных сетях.
Кроме того, результаты диссертационной работы использованы при выполнении научно-исследовательской работы №1261-1 <<Методи підвищення продуктивності безпроводових мереж наступного покоління>>.
Также полученные результаты были использованы для написания раздела 5.12 книги <<Методы научных исследований в телекоммуникациях>> \cite{popovski_method}.
Результаты исследований по повышению качества передачи речевого трафика в пакетных сетях использованы в учебном процессе кафедры телекоммуникационных систем Харьковского национального университета радиоэлектроники,
в частности, в дисциплине ''Мобильные системы связи`` при выполнении лабораторных работ.



\textbf{Личный вклад соискателя.} В статьях, выполненных в соавторстве, лично автору принадлежат следующие результаты: 

В работе \cite{my1} автору принадлежит синтез алгоритма оценки сетевой задержки.

В работе \cite{my2} автору принадлежит разработка алгоритма оценки джиттера на основе рекурсивных фильтров.

В работе \cite{my4} автору принадлежит анализ эффективности использования робастного фильтра Калмана для оценки процесса задержки.

\textbf{Апробация результатов диссертации} проводилась в ходе четырех научно-технических конференций и форумов:
14-й Международный молодежный форум ''Радиоэлектроника и молодежь в ХХІ веке``;
15-ый международный молодежный форум ''Радиоэлектроника и молодежь в XXI веке``;
4-ый международный форум ''Прикладная радиоэлектроника. Состояние и перспективы развития`` (МРФ 2011);
12th International Conference ''Experience of Designing and Application of CAD Systems in Microelectronics (CADSM)``.



\textbf{Публикации.} Основные результаты по теме диссертации изложены в статьях~\cite{my1,my2,my3,my4,my5} в специализированных научных изданиях, утвержденных в ВАК~Украины, Кроме того, материалы диссертации опубликованы в 4 тезисах доклада на научно-технических конференциях и форумах \cite{my6,my7,my8,my9} и в одном отчете по НИР.


\clearpage