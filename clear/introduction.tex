\chapter*{ВВЕДЕНИЕ}							% Заголовок

\textbf{Актуальность темы.} Согласно прогнозу \cite{ericsson} 60\% людей к концу 2018 года будут иметь покрытие LTE. Как мы видим беспроводные технологии все больше внедряются в наш повседневный мир. Такие техологии как WiFi, Bluetooth, GSM стали уже неотъемлемой частью нашей жизни. 
Современные перспективные мобильные сети развиваются в направлении внедрения следующего поколения NGN (Next Generation Network). На данный момент основными представителями таких сетей являются WiMAX и LTE сети.

Основными преимуществами использования стандарта LTE является то, что сети построенные на его основе, оптимизированы для передачи данных и реализованы в виде коммутации пакетов.
LTE не включает в себя домен коммутации пакетов для предостовления услуг передачи речи.

Спрос на услуги мобильного широкополосного доступа растет и операторы запускаю высоко скоростные сети на основе LTE. 
Тем не менее, услуги передачи речи приносят около 70\% общего дохода операторов и ясно, что эта функциональность должна быть реализована и в сетях LTE.

Сети LTE работают на стыке проводной и беспроводной сети. Практика показывает, что основные потери качественных характиристик обслуживания (QoS) происходят на границе различных сред передачи.
При передачи мультимедийной информации по комбинированным сетям с различными технологиями передачи данных, важным является выполнение требований к качеству предостовления мультимедийной информации пользователю.
При этом для трафика реального времени такого, как трафик VoIP и видео звонков, важными являются такие сетевые характеристики:
задержка, число потерянных и поврежденных пакетов и джиттер задержки.
Согласно \cite{rokovoy} наибольший вклад в задержку и потери пакетов вносит не оптимальный буфер компенсации джиттера (буфер воспроизведения). До 40\% допустимой задержки, определенной в рекомендации \cite{G114}, может состовлять задержка буфера компенсации джиттера.
Еще одной проблемой является то, что оконечные устройства могут компенсировать ограниченный размер джиттера (порядка~50~мс).

Следовательно, актуальной является научная задача, которая состоит в разработке методов мониторинга текущего сетевого метода предварительной компенсации джиттера на границах проводной и беспроводной сети.


\textbf{Связь работы с научными программами, планами и темами.} Диссертационная работа связана с реализацией основных положений <<Концепции национальной информационной политики>>, <<Концепции Национальной программы информатизации>>, <<Основных принципов развития информационного общества в Украине на 2007 - 2015 года>> и <<Концепции конвергенции телефонных сетей и сетей с пакетной коммутацией в Украине>>.
Результаты работы использованы при выполнении научно-исследовательской работы №261-1 <<Методи підвищення продуктивності безпроводових мереж наступного покоління>>, в которых автор выступал соисполнителем.

\textbf{Цель и задача исследования} состоит в повышении качества обслуживания в гибридных сетях, которые содержат мобильную и стационарную компоненту.

В ходе решения научной задачи сформулированы и решены частные задачи исследования:
\begin{enumerate}
  \item Провести анализ статистических характеристик джиттера в стационарных и беспроводных сетях.
  \item Определить основные механизмы влияния на параметы джиттера.
  \item Определить статистические нестационарности джиттера и произвести классификацию нестационарных явлений задержки.
  \item Обосновать и разработать математическую модель джиттера, позволяющую отображать динамику изменений состояний.
  \item Разработать алгоритмы статистической оценки параметров джиттера и управления с целью его минимизации.
  \item Разработать практические предложения по выбору параметров и мест установки агента минимизации джиттера на границе стационарной и мобильной сети.
\end{enumerate}

{\itshape Объект исследования:} процесс передачи трафика реального времени через гибридные сети.

{\itshape Предмет исследования:} математическая модель джиттера, модель буфера компенсации джиттера.

{\itshape Методы исследования.} В ходе разработки алгоритма статистической оценки параметров джиттера был использован робастный фильтр Калмана. Для разработки математической модели джиттера был использован аппарат теории выбросов. В ходе проведения оценки эффективности использовались методы имитационного моделирования.


\textbf{Научная новизна полученных результатов.} 
\begin{enumerate}
  \item Впервые проведен анализ условий возникновения джиттера в гибридных сетях, который включает анализ основных причин в стационарных и мобильных сетях .
  \item Разработана математическая модель задержки прибытия пакетов, научная новизна которой состоит в том, что в ней в отличии от известых моделей, учитывают различные флуктуации задержки (скачки и выбросы), которые не соответствуют нормальному распределению.
  \item Разработан инвариантный алгоритм буфера компенсации джиттера, который позволяет решить ряд проблем возникающих, когда процесс пакетной задержки отклоняется от нормального распределения и имеет выбросы и скачки.
  \item Предложены рекомендации по практическому применению буфера компенсации джиттера в сетях LTE на основе потоковых агентов.
\end{enumerate}

\textbf{Практическое значение полученных результатов.} Полученные научные результаты имеют практическое назначение, поскольку они ориентированы на дальнейшее внедрение в реальные системы связи, в частности, в диссертационных исследованиях предложен метод предварительной компенсации джиттера на границе проводной и беспроводной сети на основе потовых агентов.
Кроме того, результаты диссертационной работы использованы при выполнении научно-исследовательской работы №261-1 <<Методи підвищення продуктивності безпроводових мереж наступного покоління>>.


\textbf{Личный вклад соискателя.} В работах, выполненных в соавторстве, лично Кобрину А. В. принадлежат следующие результаты: 

В работе \cite{my1} автору принадлежит обзор алгоритма оценки сетевой задержки.

В работе \cite{my2} автору принадлежит разработка алгоритма оценки джиттера на основе рекурсивных фильтров.

В работе \cite{my4} автору принадлежит анализ эффективности использования робастного фильтра Калмана для оценки процесса задержки.

\textbf{Апробация результатов диссертации} проводилась в ходе XXX научно-технических конференций и форумов. 


\textbf{Публикации.} Основные результаты по теме диссертации изложены в статьях~\cite{my1,my2,my3,my4,my5} в специализированных научных изданиях утвержденных в ВАК~Украины, Кроме того, материалы диссертации опубликованы в XXX тезисах доклада на научно-технических конференциях и форумах [XXX-XXX] и в одном отчете по НИР.


\clearpage