\pgfplotsset{width=15cm, height=10cm, compat=1.3}
\begin{figure} [!ht]
  \center
\begin{tikzpicture}
%\pgfkeys{/pgfplots/legend pos=north west}

\begin{axis}[
mark options={scale=0.25},
ylabel={Mem [GB]},
legend style={
%area legend,
at={(0.5,-0.2)},
anchor=north,
legend columns=2},
legend cell align=left,
%cycle list name=mark list,
xlabel=Вероятность скачка задержки,
ylabel=Оценка Е-модели,
xmode = log,
xlogBase=10,
ymin=40,
cycle list name=linestyles*,
ymax=93
]
                                \addplot+[color=red,no marks, mark options={solid}] coordinates {
                                        (0.01, 80.243816584)
                                        (1e-05, 92.0889919507)
                                }; \addlegendentry{Буфер на основе ГРФК}
                                \addplot+[color=blue,no marks, mark options={solid}] coordinates {
                                        (0.01, 60.9513596189)
                                        (1e-05, 92.0889919507)
                                }; \addlegendentry{Классический буфер}

                                \addplot+[color=red,only marks,style={solid}, mark=star, mark options={solid},error bars/y dir=both,error bars/y fixed=2.7] coordinates {
                                        (0.01, 80.243816584)
                                        (0.002, 82.243816584)
                                        (0.001, 84.7696459001)
                                        (0.0002, 86.7696459001)
                                        (0.0001, 88.0889919507)
                                        (1e-05, 92.0889919507)
                                };

                                \addplot+[color=blue,only marks,style={solid}, mark options={solid},error bars/y dir=both,error bars/y fixed=2.7] coordinates {
                                        (0.01, 60.9513596189)
                                        (0.002, 65.9513596189)
                                        (0.001, 70.8931914379)
                                        (0.0002, 75.8931914379)
                                        (0.0001, 86.9536264595)
                                        (1e-05, 91.9536264595)
                                }; 

                                




\end{axis}
\end{tikzpicture}
\caption{Зависимость качества передачи речи от вероятности появления скачка задержки}
  \label{img5:qos_ver}
\end{figure}


\begin{figure} [!ht]
  \center
\begin{tikzpicture}
%\pgfkeys{/pgfplots/legend pos=north west}

\begin{axis}[
mark options={scale=0.25},
ylabel={Mem [GB]},
legend style={
%area legend,
at={(0.5,-0.2)},
anchor=north,
legend columns=2},
legend cell align=left,
%cycle list name=mark list,
xlabel=Амплитуда скачка задержки (мс),
ylabel=Оценка Е-модели,
ymin=40,
cycle list name=linestyles*,
ymax=93
]
                                \addplot+[color=red,no marks, mark options={solid}] coordinates {
                                        (5, 88.0889919507)
                                        (200, 78.243816584)
                                        
                                }; \addlegendentry{Буфер на основе ГРФК}
                                \addplot+[color=blue,no marks, mark options={solid}] coordinates {
                                        (5, 86.0889919507)
                                        (200, 72.9513596189)
                                        
                                }; \addlegendentry{Классический буфер}

                                \addplot+[color=red,only marks,style={solid}, mark=star, mark options={solid},error bars/y dir=both,error bars/y fixed=2.7] coordinates {
                                        (5, 88.0889919507)
                                        (20, 86.243816584)
                                        (50, 85.7696459001)
                                        (100, 81.7696459001)
                                        (200, 78.243816584)
                                };

                                \addplot+[color=blue,only marks,style={solid}, mark options={solid},error bars/y dir=both,error bars/y fixed=2.7] coordinates {
                                        (5, 86.0889919507)
                                        (20, 82.243816584)
                                        (50, 83.7696459001)
                                        (100, 76.7696459001)
                                        (200, 72.9513596189)
                                }; 

                                




\end{axis}
\end{tikzpicture}
\caption{Зависимость качества передачи речи от амплитуды скачка задержки}
  \label{img5:qos_amp}
\end{figure}