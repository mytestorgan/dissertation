\chapter*{ЗАКЛЮЧЕНИЕ}
Современные телекоммуникационные системы представляют собой сложную распределенную автоматически функционирующую структуру обладающую требованиями концепции Глобальной информационной системы.
Вместе с тем, остается много неисследованных или слабо исследованных научных и технологических задач, решение которых являются актуальными уже после внедрения той или иной технологии.

Так оказалась нерешенной задача надежного функционирования в гибридных сетях, состоящих из мобильной и стационарной компоненты.
В частности, появление джиттера недопустимо возрастает по какой либо причине не зависимо возникающей в одной и в другой сети.

В результате подготовки диссертационных исследований была поставлена и решена научная задача разработки метода повышения качества обслуживания на стыке мобильной и стационарной сети на основе оптимизации потокового агента.
В результате решения этой задачи получены следующие выводы:
\begin{enumerate}
 \item В результате анализа характерных режимов работы существующих методов компенсации джиттера оказалось, что классические методы имеют значительные ошибки оценки так, как 
 статистика процессов задержки пакетов в гибридных сетях существенно нестационарна в связи с появлением в произвольные моменты времени выбросов и скачков задержки.
 При этом значительно возрастает априорная неопределенность статистики джиттера.
 
 \item Случайные изменения джиттера могут быть описаны случайным законом с нормальным распределением вероятностей.
 Нормальность закона обуславливается множеством причин формирующих эту случайность, что дает основания использовать результаты центральной предельной теоремы.
 Нормальный закон изменения джиттера не является однородным, а представляет собой засоренную модель.
 К факторам засорения следует отнести наличие скачков и выбросов задержки.
 Наличие указанных засоренностей приводит к соответствующим ошибкам при применении классических методов обработки.
 
 \item В работе предложено две различные математические модели задержки пакетов в гибридных сетях.
 Первая модель представляет собой джиттер в виде случайной величины и может быть использована для стационарных условий работы сети.
 В условиях нестационарности целесообразно использовать модели, представленные в виде пространства состояний, 
 что дает возможность рассмотрения динамического процесса, адаптации модели к изменению статистики, получению рекурсивных алгоритмов оценки задержки и управления компенсацией джиттера задержки.
 
 
 \item В условиях априорной неопределенности для обработки джиттера (оценки и управления) применение параметрических методов оказывается нежелательным и неэффективным, поскольку при этом решение связано с большой размерностью решаемой задачи, потерей устойчивости и снижением качества оценки.
 Более приемлемым является использование робастных непараметрических методов, которые при прочих равных условиях не уступают параметрическим методам.
 
 \item Разработаны и предложены методы формализации наблюдения джиттера
 на основе алгоритма Хьюберта, что дало возможность представления засоренного процесса в одном алгоритме, включая как выбросы, так и скачки.
 
 \item С использование теоремы о разделении синтезировано 2 метода управления на основе управления наблюдением.
 Алгоритмы управление реализованы на основе оценки комплексного вектора весового коэффициента, который обеспечивает коррекцию фазы в методе управления буфером компенсации джиттера.
 

 
 \item В соответствии с предложенными моделями разработан метод оценки джиттера для первой модели, представленной в виде случайной величины, 
 c использованием робастных алгоритмов на основе функции правдоподобия.
 Для математической модели, представленной в виде пространства состояний, предложено использовать робастный рекурсивный алгоритм Калмана-Бьюси.
 Сравнительный анализ показал, что в условиях наличия скачков и выбросов у классических алгоритмов имеет место значительные ошибки, в среднем превосходящие на $20-25\%$ ошибки робастных алгоритмов.
 В тоже время, время сходимости робастных алгоритмов на порядок меньше, чем у классических методов.
 
 \item Выработана рекомендация по практическому применению разработанного метода компенсации джиттера в сетях LTE.
 В качестве платформы для внедрения буфера компенсации джиттера предложено использовать потоковые агенты, размещенные на границе между проводной и беспроводной сетью.
 Основными преимуществами данного предложения является:
 \begin{itemize}
  \item Повышение качества передачи речи в гибридных сетях.
  \item Выполнение предварительной компенсации джиттера в гибридных сетях и тем самым упрощение задачи буфера на оконечных устройствах.
  \item Внедрение концепции потоковых агентов в сети LTE позволяет использовать другие полезные функции для мультимедийного трафика.
 \end{itemize}
 
\end{enumerate}
